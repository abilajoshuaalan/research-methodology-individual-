\documentclass[a4paper,12pt]{article}
\begin{document}


\begin{Huge}
\begin{center}
\begin{normalsize}

\textbf{MAKERERE UNIVERSITY } \\
\textbf{FACULTY OF COMPUTING AND INFORMATICS TECHNOLOGY} \\
\textbf{SCHOOL OF COMPUTING AND INFORMATICS TECHNOLOGY} \\
\textbf{DEPARTMENT OF COMPUTER SCIENCE} \\
\textbf{BACHELOR OF SCIENCE IN COMPUTER SCIENCE} \\
\textbf{BIT 2207 RESEARCH METHODOLOGY} \\
\textbf{YEAR 2} \\
\textbf{\sc ABILA JOSHUA } \\
\textbf{\sc Reg No: 16/U/19130/PS } \\
\textbf{\sc std No: 216021700}\\
\textbf{\sc Course Work: Assignment 1}\\
\end{normalsize}
\end{center}
\end{Huge}
\newpage

\section{\sc ugandan footballers failing to make an international scene}
\paragraph{\sl Most Ugandan football fans exhibit a lot of loyalty to top European soccer clubs, even when leading teams like Manchester United, Arsenal, Real Madrid, Barcelona and Bayern Munich do not have a Ugandan footballer in their ranks. This begs the question: To what extent would this loyalty manifest if Real Madrid signed Vipers’ Erisa Ssekisambu?}
\paragraph{\sl Well, there are numerous examples of players who’ve played in the local league and successfully made it to the national senior team, the Uganda Cranes. The players have a lot of national team experience and most of them dream of playing professional football in Europe.}
\subsection{\sc Football in Uganda}
\paragraph{ \sl However, the only Ugandan footballer to have risen to a team in the big five European leagues: the Barclays Premier League, Spanish La Liga, French Ligue 1, German Bundesliga and Italian Serie A, remains Majid Musisi (RIP) who played for French club Stade Rennes}
\paragraph{\sl Ibrahim Ssekagya of Red Bull Salzburg in Austria’s Bundesliga and David Obua who featured for the Scottish Premier side Hearts of Oak, both never made it to the top 5 leagues in Europe. Other Ugandans who have made it to Europe include Nestroy Kizito, Eugene  Ssepuya, Tony Mawejje, Andy Mwesigwa, Noah Kasule, among others but these have not gone beyond mediocre teams like FK Vojvodna in Serbia, FC Cuckarick (Serbia), IBV Vestmaneyer in Iceland.}
\subsection{\sc causes of failure}
\paragraph{\sl Indeed, most of them have failed to make a breakthrough to the top European leagues, leveraging their play on the African continent, after which they end up making a U-turn back to the home based league within a short period; they end up at their former clubs due to numerous factors, both at home and where they attempted professional football}
\paragraph{\sl To get to the cause of the problem, one needs to look at the nature of competition in the local league that would help produce the talent required for top world leagues, as well as poor administration by the local football federations, and the inadequacy of funds, among other issues.
Inexperienced player agents also suggest local players are poorly-marketed; players are not exposed to the international scene through advertising them on the internet by displaying their videos on platforms like YouTube. Agents have got more control over the player and some others involve in bribery.
}
\paragraph{\sl The absence of strong club foundations has also not helped develop talent, and most clubs don’t follow up the players after selling them to foreign clubs. The exception is Vipers FC, one of the few clubs that has got a foundation for its players, grooming them from St. Mary’s Kitende for competitive football in the league as well as helping them with transfers.}
\paragraph {\sl Language is another barrier that affects most Ugandan footballers; some are poor at English and prefer verbal agreements with the clubs. A written agreement is commonly used by most clubs in this era and players find problems negotiating contract terms with the club because they don’t like to read lengthy articles in the contract.}
\paragraph{\sl Talking Tactics: Football is played in different styles in most countries, and almost every country believes in its type of football for success. Most footballers will find football played in some countries difficult to adapt to}
\paragraph{\sl Weather conditions: Unfavorable conditions like hot temperatures affect some players’ form because they fail to adapt. This makes up a player’s mind and finds conditions at home easy for him to live in.}
\paragraph{\sl Lack of football education and motivation from football federations are other factors that have miserably failed Ugandan footballers to make it to the international level.
Players that keep on coming back from professional stints to the local clubs they played for before at home not only diminishes their profiles but also sends them into oblivion
}
\section{\sc types of research used}
\paragraph{\sl Conceptual research ,this research involved a conceptual aspect  because  the idea discussed was gathered from a series of  existing ideas or theories so as to explain events or incidences as they occur.}
\subsection{\sc references}

\textbf{\sl 	
Ref: http://eagle.co.ug/2016/07/05/ugandan-footballers-failed-make-international-scene.html
}



\end{document}